\chapter{Conclusions}~\label{ch6}

% The driven states $(x_{n-1},x_n)$ hold the information of the left-infinite history of the input. Furthermore, there is some evidence that the inverse-limit space (not system) seems to distinguish two dynamical systems that are not topologically conjugate for a class of systems in the sense their inverse-limit spaces are not homeomorphic to each other.  So the relation $Y_T$, which is homeomorphic to the inverse-limit system, may be unique to a class of systems. (The reference is a paper on Ingram’s Conjecture by Marcy Barge, Henk Bruin and Sonja Stimac.) Further investigation is needed.  \cite{budivsic2012applied, korda2020data}

There are a myriad of techniques to generate accurate models for observed dynamical systems using data obtained from measurement functions.
Core to the success of such methods is the fact that we can often characterise the dynamics from am observed space by a mapping of much lower functional complexity when compared to the true map describing the unknown underlying system. 
Takens theorem and the numerous generalizations thereof is one such method establishing (under some generic conditions) the learnability of a system produced by creating a delay-coordinate vector from a sufficiently large number of previous observations from a dynamical system. 
In this case the new system is topologically conjugate to the system from which the observed time-series was obtained.
A critical concern of such methods is the requirement of full knowledge of the underlying state space variables governing the dynamical system one is interested in. In real-world conditions this information is rarely available and, when it is, is often contingent upon human expertise and insight. 

Each discrete-time state space model has at its core a driven dynamical system. 
In this project, it has been shown that an input originating from a compact space entailing exactly one solution (i.e., if the driven system has the USP) is equivalent to representing information without distortion, a concept which was formalised by establishing the existence of a causal mapping’s. 
The established results are general, implying that any driven dynamical system with specific properties, such SI-invertibility, can give rise to a causal mapping. 
In particular, we have a topological conjugacy (semi-conjugacy) between the data comprising the single-delay lag dynamics of a driven system~\cite{manjunath2013echo} and the underlying homeomorphic dynamical system. 
Finite-length input data is shown to be sufficient for forecasting as the left-infinite segment of the past of the input beyond some arbitrarily large but finite time (a result which holds due to the universal semi-conjugacy of a driven system’s continuity) influences the driven system's states in an almost negligibly small manner.

Viewed through the lens of reservoir computing, we have also demonstrated that when single-delay dynamics are utilised, a learnable map in a reservoir computing network certainly exists. 
Moreover, within a reservoir computing network, such a driven system posssesses certain proven qualities pertaining to the robustness of the system to different forms of perturbation. 
We are thus able to preserve the quality of a representation that would otherwise have remained sensitive to various fluctuations and rendering the learning of the system’s dynamics wholly unpredictable. 


Finally, we have considered some physical systems to observe the methodology in practice and verify that we can indeed predict the future evolution of chaotic systems such as the double pendulum.
We demonstrated the robustness of our methodology to external noise by considering the cases of the  Dream Attractor and the Double Pendulum.

\ednote{Loose paragraph commented out}
% Considered from a philosophical-scientific  perspective, this project has aimed to establish a juncture between rationalism (the theory of nonautonomous dynamical systems) and empiricism (machine learning) to yield results profitable to the discipline of Applied Mathematics. 
% This was done by showing that dynamics in an observed space can be established as topologically conjugate to the underlying system. 
% As a result of the topological conjugacy, deep learning techniques are then subsequently applied to physically such equations from the data. 
% In so doing, we acquire models of high fidelity that provide exceptional forecasting results on chaotic data that exhibit both statistical and topological consistency in the long run.
