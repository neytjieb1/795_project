\begin{abstract}
    It is a well-established practice to collect measurements from underlying artificial, natural, and physical systems through observables to realise a dynamical model with greater descriptive power. 
    Many approaches to forecasting exist but they necessitate an understanding of systems we often don't know much about.
  
    In this project, the base assumption is that data originates from  some discrete-time dynamical system, particularly one of a chaotic nature, 
    but no assumptions are made pertaining to a specific dynamical system.
    A general learning problem is formulated and a well-known approach to the problem, alongside its practical limitations, is then considered.
    The notion of driven dynamical systems is discussed, and we establish the theoretical conditions that guarantee its single-delay dynamics to be topologically conjugate (`dynamically equivalent') to the dynamics of the underlying system from which measurements were originally taken. 
    The methodology is implemented practically by making use of recurrent neural networks and deep learning methods to learn a map that in turn helps forecast the future observations of the underlying system.
    In so doing we present the workability of an approach to obtain long-term topological and statistically consistent predictions for some chaotic attractors, and notably the double pendulum -- a system difficult to model from data.
    \end{abstract}
  