\begin{abstract}
    It is a well-established practice to collect measurements from underlying artificial, natural, and physical systems through observables in order to realise a dynamical model with greater descriptive power. 
    Many approaches to forecasting exist but they necessitate an understanding of systems we often don't know much about.
  
    In this project, the base assumption is that data is originates from  some discrete-time dynamical system, particularly one of a chaotic nature, but no assumptions are made pertaining to a specific dynamical system.
    A general learning problem is formulated and a well-known approach to the problem, alongside its practical limitations, is then considered.
    The notion of driven dynamical systems is introduced, and we subsequently establish the theoretical existence of a dynamical system possessing certain properties which guarantees it to be topologically conjugate (`dynamically equivalent') to the dynamics of the underlying system from which measurements were originally taken. 
    Our methodology is implemented practically by making use of recurrent neural networks to learn a map topologically equivalent to the underlying system that forecasts future states.
    In so doing we present the workability of an approach to obtain long-term topological and statistically consistent predictions for simple physical systems as well as a some chaotic attractors.
    \end{abstract}
  