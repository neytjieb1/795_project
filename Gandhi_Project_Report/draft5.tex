\documentclass[a4paper,12pt,twoside]{report}
% \documentclass[a4paper,12pt,twoside]{book}
\pagestyle{headings}
\usepackage[utf8]{inputenc}
\usepackage{filecontents}
\usepackage[many]{tcolorbox}
\usepackage{graphicx}
\usepackage{harpoon}
%\usepackage{apacite} 
\usepackage{xcolor}
\usepackage{latexsym}
\usepackage{amssymb}
\usepackage{amsmath}
\usepackage{mdframed}
%\usepackage[authoryear]{natbib}
\usepackage{hyperref, url}
\usepackage[show]{ed}
\usepackage{subfigure}
\usepackage[ntheorem]{empheq} 
\usepackage{amsthm, amssymb, amsfonts, latexsym}
% \usepackage{pst-all}

\usepackage{enumitem}
\usepackage{multirow}
\usepackage{longtable}
\usepackage[export]{adjustbox}

\setlength{\parskip}{1.7em}
\setlength{\parindent}{0em}


%\setlength{\baselineskip}{20pt}

\setlength{\textwidth}{429.75499pt}
\setlength{\textheight}{643.20255pt}
\setlength{\oddsidemargin}{5 mm}
\setlength{\evensidemargin}{5 mm}
\setlength{\topmargin}{0 mm}
\setlength{\headsep}{0 mm}
\setlength{\headheight}{0 mm}


\usepackage{pst-node}

\newcommand{\citealtt}[1]{\citeauthor{#1},\citeyear{#1}}
\newcommand{\myycite}[1]{\citep{#1}}

\mathchardef\mhyphen="2D

%\newcommand{\cev}[1]{\reflectbox{\ensuremath{\vec{\reflectbox{\ensuremath{#1}}}}}}


\usepackage{pst-node,graphicx,pst-blur}
%\uspackage{auto-pst-pdf}
%\usepackage{tikz-cd} 



\makeatletter
\DeclareRobustCommand{\cev}[1]{%
  \mathpalette\do@cev{#1}%
}
\newcommand{\do@cev}[2]{%
  \fix@cev{#1}{+}%
  \reflectbox{$\m@th#1\vec{\reflectbox{$\fix@cev{#1}{-}\m@th#1#2\fix@cev{#1}{+}$}}$}%
  \fix@cev{#1}{-}%
}
\newcommand{\fix@cev}[2]{%
  \ifx#1\displaystyle
    \mkern#20mu
  \else
    \ifx#1\textstyle
      \mkern#20mu
    \else
      \ifx#1\scriptstyle
        \mkern#26mu
      \else
        \mkern#26mu
      \fi
    \fi
  \fi
}

\makeatother

\newcommand{\xcm}{\epsfxsize=3.1cm}
\newcommand{\fig}[1]{\epsfbox}
% \newcommand{\bp}{\begin{minipage}{3.1cm}}
% \newcommand{\ep}{\end{minipage}}



\newtheorem{Definition}{Definition}[]
%\newmdtheoremenv{Theorem}{Theorem}[]
\newtheorem{Theorem}{Theorem}[]
\newtheorem{Lemma}{Lemma}[]
\newtheorem{Proposition}{Proposition}[]
\newtheorem{Corollary}{Corollary}[]
\newtheorem{Remark}{Remark}[]
\newtheorem{Example}{Example}[]


%Shortcut symbols
\newcommand{\Ftheta}{\ensuremath{F_\theta}}


\makeatletter 
\renewcommand{\thefigure}{\@arabic\c@figure}
\makeatother
\usepackage{xr}


%%%% PNAS GRAPHICS SETUP- FROM THE PNAS STYLE LATEX FILE
\RequirePackage{graphicx,xcolor}
\RequirePackage{colortbl}
\RequirePackage{booktabs}
\RequirePackage{algorithm}
\RequirePackage[noend]{algpseudocode}
\RequirePackage{changepage}
\RequirePackage[twoside,%
				letterpaper,includeheadfoot,%
				layoutsize={8.125in,10.875in},%
                layouthoffset=0.1875in,%
                layoutvoffset=0.0625in,%
                left=38.5pt,%
                right=43pt,%
                top=43pt,% 10pt provided by headsep
                bottom=32pt,%
                headheight=0pt,% No Header
                headsep=30pt,%
                footskip=25pt,
                marginparwidth=38pt]{geometry}
\RequirePackage[labelfont={bf,sf},%
                labelsep=period,%
                figurename=Fig. ]{caption}
\setlength{\columnsep}{13.5pt} % Distance between the two columns of text
\setlength{\parindent}{12pt} % Paragraph indent

%% Figure caption style
\DeclareCaptionFormat{pnasformat}{\normalfont\sffamily\fontsize{7}{9}\selectfont#1#2#3}
\captionsetup*{format=pnasformat}

%%% GREYBOX AROUND FIG
\definecolor{lightgray}{gray}{0.95}
\newcommand\greybox[1]{%
  \vskip\baselineskip%
  \par\noindent\colorbox{lightgray}{%
    \begin{minipage}{\textwidth}#1\end{minipage}%
  }%
  \vskip\baselineskip%
}

\newtcolorbox{paperbox}[1][]{
    enhanced,
    colback=white,
    boxrule=0pt,
    boxsep=0pt,
    arc=0mm,
    width=0.8\linewidth,
    fuzzy shadow={0mm}{-4pt}{-4pt}{1mm}{black!30!white},
    #1
}


% \externaldocument{Supp_Koopman-0808}
\begin{document}


\title{Data-Driven Modelling \& \\ Forecasting of Chaotic Systems}



\author{B.L. Nortier} 
\date{January, 2022}
\date{}
\maketitle
% \tableofcontents


\chapter{Introduction}\label{ch1}

Experimenting on biological, physical, and artificial systems in order to generate a more informative dynamical model is a well-established practice in modern science. Traditional methods for modelling physical systems are based on laws of physics that are based either on empirical relationships or on intuition. For systems that evolve with time, physical laws yield mathematical equations that govern how the quantities evolve with time. However our world around is, for the most part, much more complex than that which can be distilled into elegant equations. We do not fully understand many of the more complex systems, nor do they even provide us with a good physical intuition of the underlying principles governing the dynamics. 
To complicate this - the underlying systems we observe often display sensitive dependence on initial conditions; despite having highly similar initial conditions, differing orbits diverge quickly and to such an extent that it becomes seemingly impossible to retrace their steps back to the original conditions. Even the presence of usually 'negligible' computational noise or measurement error renders long-term point-wise prediction infeasible. 
There are many difficulties that one encounters while modelling such systems:
\vspace{-8mm}
\begin{enumerate}[noitemsep, label=\roman*.]
  \item One may not have access to the complete states of the systems
  \item The actual system may be described by functions that behave wildly in the sense their graphs have a wild oscillatory behaviour, i.e., they have a large functional complexity.    
\end{enumerate}


Models derived primarily from data can be classified into three categories: 
\vspace{-8mm}
\begin{enumerate}[noitemsep, label=\roman*.]
  \item  Interpretable models (i.e., they establish relationships between internal physical mechanisms), 
  \item Partially interpretable models capturing some modes of the dynamics, 
  \item Non-Interpretable models, defined as such mainly due to the fact that they are defined on a different phase space that is usually high-dimensional. 
\end{enumerate}

Examples in the literature attempting to forecast data from such systems have tried a number of different approaches with differing degrees of success. 
When we have complete access to states of the system, an ordinary differential equation can be obtained from data (\cite{brunton2016discovering, champion2019data} and \cite{small2002modeling,xu2006modeling}) wherein one could approximate the vector field by a library of functions to obtain interpretable models. 


Recent partially interpretable models available in the literature have been based on the Koopman operator (e.g.,\cite{koopman1932dynamical,budivsic2012applied}) to  employ observables mapping the data onto a higher-dimensional space. This then makes the dynamics in the higher-dimensional space more amenable for approximation by a linear transformation. Such methods, obviously, do not guarantee an exact reconstructions for nonlinear models, and in practice provide poor long-term consistency for a large class of systems. \textbf{(Citation)}

The non-interpretable models include the delay embedding and the machine learning algorithms. One could learn a system conjugate to the underlying system by applying the Takens delay embedding \cite{takens1981detecting} when one has ``good" observations from the system; this learnt system could then be used for forecast the observed data.  Takens delay embedding theorem \cite{takens1981detecting} and its various generalisations (e.g., \cite{sauer1991embedology, stark1999delay, gutman2018embedding}) establishes the learnability of a system constructed by concatenating sufficiently large previous time-series observations of a dynamical system into a vector (called delay coordinates). This then establishes the existence of a map on the space of delay coordinates equivalent (or topologically conjugate) to the underlying map from which the observed time-series was first obtained. Although topological conjugacy guarantees an alternate representation of the underlying system, the quality of this representation still depends on numerous parameters, making the comprehension of the dynamics rather unreliable (e.g., \cite{principe1992prediction}). One reason for this fragility is that the embedded attractor in the reconstruction space not always an attractor of the map that is learnt in the reconstruction space, despite  unquestionably being an invariant set. When the embedded object is not explicitly known to be an attractor (as explained in Chapter \ref{ch3}) it can cause predictions to fail.


Practically, the application of Takens embedding involves learning a map through some technique and consequently one wishes that these would have low functional complexity\cite{manjunath2021universal}, i.e., functions with fewer oscillatory graphs. Pure machine learning methodology that processes temporal information (like the echo state networks (Citations to echo state networks)) by mapping data onto a higher dimensional space for further processing. Although they perform  well on forecasting some dynamical systems, they fail completely on others (Citations here) as there is often no guarantee that the right function was learnt during training.

This thesis deals with the implementation and analysis of non-interpretable models that can guarantee exact reconstruction. The project work concerns the study and implementation of a method (\cite{manjunath2021universal}) that incorporates learning a function by mapping the data on to a higher dimensional space using what is called a driven dynamical system (See Chapter 4). With a clear understanding of how the data is mapped onto the higher dimensional space, the method then permits the learning of a dynamical system topologically conjugate to that of the underlying system. Instead of linear regression as in the training of echo state networks, deep learning methods are employed to learn the correct function. With slight modifications to the implementation in the paper (\cite{manjunath2021universal}), we show that one can construct accurate non-interpretable models with the ability to reconstruct attractors from more hard-to-forecast systems like the double pendulum. (The forecasting of the time-series from a double pendulum has not been reported before.) Moreover, we also demonstrate that long-term statistical consistency is preserved.

\emph{By solidifying the mathematical underpinnings of our theory, we hope to guarantee the ability to construct models with predictive power ranging from molecular biology to neuroscience. in the near future. (We can modify this sentence when the report is completed.)}

The more intricate details of proofs are referred to where relevant throughout. This report is organised as follows: 
\newline In Chapter~\ref{ch2}, we recall the definition of a discrete-time dynamical system, how a discrete-system arises from a flow of an ODE and then proceed to define the inverse limit space and topological conjugacy of autonomous systems. 
\newline In Chapter~\ref{ch3}, we introduce the problem of forecasting dynamical systems, state the Takens delay embedding theorem and discuss various issues faced while forecasting. 
\newline In Chapter~\ref{ch4}, a driven dynamical system is defined and discuss the properties of a specific class of driven dynamical systems that we make use of in this thesis/report/paper. 
\newline Finally in Chapter~\ref{ch5} we show the implementation of these forecasting methods, and conclusions are provided in Chapter~\ref{ch6}.

\chapter{Discrete-time Dynamical Systems}\label{ch2}

In this chapter, we provide a brief description of what a discrete-time dynamical system is and what it means for it to exhibit chaos. We refer to \cite{devaney2018introduction, de2013elements} for more details. 

At its most elementary level, a dynamical system is just something that evolves deterministically through time. In the context of this thesis, deterministic refers to the fact that a system evolves according to specified rules rather than based on random events. 
Dynamical systems arise in a variety of situations. A continuous-time dynamical system describes the states for all values of the time. Specifically, if the motion of a pendulum in which the quantities such as the angular position and angular momentum are known at all times, then it is a continuous-time dynamical system. The equations of the dynamical system can take the form of one or more ordinary differential equations that determine the relevant quantities at any future time if we know the initial location and momentum. 
In ecology, discrete-time dynamical systems are widely used to model population growth. The model in this case is a function calculating the following generation's population given the population of the previous generation. If we know the starting population, we may once again calculate the population at any time in the future. 

Formally, a function $T: U \to U$, where $U$ is some set is  a \emph{discrete-time dynamical system} and its iterates $\{u,Tu,T^2u,\ldots\}$, where $T^n$ denotes the $n$-fold composition of $T$ with itself, describe the evolution of an initial condition $u\in U$ (Note that we frequently drop the brackets and denote $T(u)$ by $Tu$ so as to simplify notations).  

Continuous-time dynamical systems modelled using ordinary differential equations can give rise to discrete-time dynamical systems. To see this, consider a differential equation $\dot{x} = f(x)$, $f: \mathbb{R}^n \to \mathbb{R}^n$, $n\in\mathbb{N}$ given to have a unique solution passing through 
each point $x\in\mathbb{R}^{n}$, call it $x(t)$ where $x(0)=x_0$. 

\begin{Definition}
  [\bf Flow of an Equation] \label{Dfn_Flow}\rm
  The flow of the equation  $\dot{x} = f(x)$  is defined to be a mapping $\varphi: \mathbb{R}^n \times \mathbb{R} \to \mathbb{R}$ where $\varphi(x_0,t)= x(t)$ for the solution $x(t)$ with $x(0)=x_0$ 
\end{Definition}

By fixing $t=K \in (0,\infty)$, we can define the \emph{time-$K$ map} as  $T(x):= \varphi(x_0,K)$, and it is easily verified that $T\circ T(x_0) = \varphi(x_0,2K)$. In general the $m^{\mbox{th}}$ iterate of $x_0$ under $T$ would be the value of the solution of the ODE evaluated at time $mK$ with the initial condition $x_0$, i.e. $\varphi(x_0, mK)$. 
Thus ordinary differential equations give rise to a discrete-time dynamical system by sampling the value of the solution $x(t)$ at time intervals $K$ units apart. 


A numerical discretization of a differential equation can also give rise to a discrete-time dynamical system. For instance, Euler's method approximates $\dot{x}(t)$ by $(x(t+h)-x(t))/h$; if $h$ is fixed throughout, the solution of a differential equation $\dot{x}=f(x)$ 
can be approximated at the time instant $t+(m+1)h$ by iterating the equation 
\begin{equation}
  x(t+(m+1)h) = x(t+mh) + h f(x(t+mh))
\end{equation}

Adopting more succinct notation by replacing $x(t+mh)$ with $u_m$, we rewrite the above equation as
\begin{equation}
u_{m+1} = u_m + hf(u_m)
\end{equation}
or in even simpler terms as the discrete-time dynamical system with map $T(u) = u + hf(u)$, where $T$, $u$ and $f$ are understood to be as above.


Of course, discrete-time dynamical systems need not always arise through a differential equation, and once again we may consider the field of ecology where discrete dynamical systems are often directly derived or assumed. \textbf{(Citation)}
There is a school of thought that advocates discrete-time dynamical systems to be more natural for modelling real-world observations than differential equations and we refer the interested reader to \cite{saber2010introduction}.


\section{Invariant Sets}

\section{Chaos}

\section{Conjugacy}

\chapter{The Learning Problem}\label{ch3}

In this chapter we acquaint ourselves with the question of forecasting dynamical systems with unknown underlying structures, state and discuss the shortcomings of the Takens delay embedding theorem and discuss various issues faced while forecasting. 

Consider a relatively simple learning problem: 
Given the sequence $(u_0, u_1, \ldots, u_m)$ for $m\in\mathbb{Z}$, a finite segment of an orbit of the map $T$ where $T$ is defined by the update equation $u_{n+1} = Tu_n$, forecast the values $u_{m+1}, u_{m+1}$ where the map $T$ is unknown, given that $u_m$. 

A practical example of this would be the subsequent scenario: given the time-sequential coordinates of an object moving in space, predict the future positions of that object. However we are very rarely, if at all, presented with a problem where the entire state information is available to us in the form $u_n$ at some time-step $n\in\mathbb{N}$. Consider then a more involved learning problem:

Suppose we only have the observations $\theta(u_0), \theta(u_1), \ldots, \theta(u_m)$ of the true system states $u$ in an unknown dynamical system $(U,T)$ and we wish to predict the values $\theta(u_{m+1}), \theta(u_{m+2})$ and so forth.

First we establish the method of Takens delay embedding. 
In this method, one considers a discrete-time dynamical system defined on a 'nice' space (a smooth manifold) that can be obtained as time-$K$ map of a continuous time-dynamical system, concepts which will be formalised below.
We make observations from such a system, i.e., we observe the evolution of $\theta(w_0)$ (where $w_0$ represents the initial state) i.e., we observe  a finite set of values of $u_n :=  \theta(w_n) = \theta(Tw_{n-1})$. The sequence $\{u_n\}$ represents a scalar time-series and intuitively $\theta$ may be thought to represent a probe inserted into a larger system which is itself only measuring/extracting a small part of the greater system state at time $t=n$. 
Consider for example a thermometer erected to measure the ambient temperature in a local village. This measurement function, the thermometer, is capturing only a single aspect, the temperature, of a much grander dynamical system entailing the present weather of the surrounding area. Even more than that though- it is measuring a miniscule part of the global weather system. 


%Given the sequence $\{u_1, u_2, \ldots, u_n}$, we define the variable \newline $y_n:=[\theta_n, \theta_{n-\tau}, \ldots, \theta_{n-(2d-2)\tau}, \theta_{n-(2d-1)\tau}]^{T}\in\mathbb{R}_{d}\times{1}$ where $\tau$ represents the lag and $d$ the dimension of the attractor. (We shall return to these terms shortly and their discussion may be put on hold for the time being).

However, the true state $u$ of a system is seldom, if ever, fully known. In almost all cases we can at most insert probes into a system to obtain partial information by means of the measurements taken. Moreover, the process of taking a measurement itself introduces 2 additional aspects which complicate the problem: 
\vspace{-8mm}
\begin{enumerate}[noitemsep, label=\roman*.]
  \item A series of measurements over a specific time-interval is inherently a discretisation procedure of the underlying continuous-time dynamical system. (Hence why we restricted our attention to discrete-time systems in the preceding section)
  \item A probe will never be fully accurate, and so the act of measurement introduces a certain measure of numerical noise/inaccuracy.
\end{enumerate}



From here we construct a multidimensional observable using the method of stacking previous observations, i.e., we create delay-coordinate map defined by
$\Phi_{k,\theta}(w) := (\theta(T^{-k}w)\ldots,\theta(T^{-1}w),\theta(w))$.  The essence of Takens theorem relates to the fact that when $k$ is sufficiently large, we can define a dynamical system on the space $\mathbb{R}^{k+1}$ whose states are $\Phi_{k,\theta}(w)$, $\Phi_{k,\theta}(Tw)$, $\Phi_{k,\theta}(T^2w)$, etc. and this dynamical system is topologically conjugate to the unknown underlying system $(W,T)$. We recall the Takens delay embedding theorem next.

%More formally we may say that in the process of measuring a system, we obtain a 1-dimensional time-series $\{\theta(u(t))\}$, where $\theta:\mathbb{R}^n\to\mathbb{R}$ is a smooth observation or measurement function. This series represents a possibly noise-augmented \ednote{M: Noise-augmented is not the right word} dataset containing information of certain parts of the original system. 

%We may then ask ourselves the question: Is it in any way possible to retain information about the system state $x(t)$ in this temporal data-series $\theta(x(t))$? The answer is easily yes if T is known. However, neither this, nor even the exact function $\theta$ is available. 

\section{Takens Embedding Theorem}

% \section{Practical Issues of Takens theorem}

% \chapter{Driven Dynamical Systems to Forecast Problems}\label{ch4}

% In this chapter, we discuss results pertaining to the mapping of temporal data obtained from a discrete-dynamical system onto a different space through the notion of a driven dynamical system. 
% We also consider the conditions a driven system should possess to avoid adding distortion to its state-space representation. We then describe the \emph{single-delay dynamics} (SDD) of the system and and finally recall conditions on the driven system so that the SDD are conjugate (or at least semi-conjugate) to the underlying system. 
% The SDD can then be used to forecast and reconstruct the underlying system. 
% % \ednote{Is this attractor assumed to exist? M:  I replaced  it with the system. Attractor is also a commonly used term since we observe data from an attractor in almost all cases.}

% \section{Nonautonomous and Driven Dynamical Systems}

% \section {Unique Solution Property}

% % \section{Next section?}

% \section{Choosing the driven system $g$}

% \section{The next step in Dynamics}

% % \section*{Summarising the discussing thus far:}

% \section{A discussion of $G_T$ }

% \section{Advantages of learning $\Gamma$} \label{sect_LearnGamma}

% \chapter{Implementation} \label{ch5}
 
% In this chapter we turn to the implementation of the function $G_T$ by discussing a method to learn the function $\Gamma$. We then indicate where this implementation differs from previous work done by (\cite{manjunath2021universal}) and empirically show that the SDD of $G_T$ are less functionally complex than $\Phi_{k, \theta}$ or $T$ by considering the Pearson correlation coefficient. 
% Finally, we present and discuss some numerical results for a focus system of this project: the double pendulum. We also consider other chaotic attractors.

% \section{Implementing $G_T$}

% \section{Changes in this project to previous work of~\cite{manjunath2021universal}}


% \section{Numerical simulations}
% % In this section we showcase numerical results pertaining to 3 attractors that were simulated using the methodology described in the previous chapters. We first consider a simple physical system.
% \subsection{Double Pendulum}
% \subsection{Alternative Attractors}
% \subsubsection{Thomas' Cyclically Symmetric Strange Attractor}
% \subsubsection{Fractal Dream Attractor}
% \section{Generalised Pearson Correlation} %\ednote{B: Alter heading.}


\chapter{Conclusion}\label{ch6}

% The driven states $(x_{n-1},x_n)$ hold the information of the left-infinite history of the input. Furthermore, there is some evidence that the inverse-limit space (not system) seems to distinguish two dynamical systems that are not topologically conjugate for a class of systems in the sense their inverse-limit spaces are not homeomorphic to each other.  So the relation $Y_T$, which is homeomorphic to the inverse-limit system, may be unique to a class of systems. (The reference is a paper on Ingram’s Conjecture by Marcy Barge, Henk Bruin and Sonja Stimac.) Further investigation is needed.  \cite{budivsic2012applied, korda2020data}

As technology has advanced over the past century, humanity’s ability to record and manipulate rich, complex data in order to produce forecasts has profoundly increased. 
Collecting measurements from underlying artificial, natural, and physical systems through observables to realise a dynamical model with greater descriptive power is well-established in present-day science.

A key to the success of these techniques is the fact that dynamics discerned from the observed space can often be characterised by a mapping of much lower functional complexity when compared to the map describing the unknown underlying system. Takens theorem and the numerous generalisations (e.g., [5, 6, 7, 8, 8]) thereof is one such method establishing (under some generic conditions) the learnability of a system produced by creating a delay-coordinate vector from a sufficiently large number of previous observations of a dynamical system. This system is topologically conjugate to the system from which the observed time-series was obtained.

A critical concern of this method is the requirement of full knowledge of the variables of the underlying state space that govern the complex dynamical system one is interested in. In real-world conditions this intelligence is seldom available and one must depend heavily on human expertise and insight. 

Each discrete-time state space model has at its core a driven dynamical system. In this project, it has been shown that an input originating from a compact space entailing exactly one solution (i.e., if the driven system has the USP) is equivalent to representing information without distortion, a concept which was formalised by establishing a causal mapping’s existence. The established results are general, implying that any driven dynamical system with specific properties can give rise to a causal mapping, and in so doing  providing an answer as to the success of arbitrary choices of driven systems in information processing applications in [4, 5].

In particular, we have a topological conjugacy (semi-conjugacy) between the data comprising the 11single-delay lag dynamics of a driven system’s and the underlying homeomorphic dynamical system. Finite-length input data is sufficient for forecasting as driven system states are influenced only in an almost negligibly small manner by the left-infinite segment of the past of the input beyond some arbitrarily large but finite time (a result which holds due to the universal semi-conjugacy of a driven system’s continuity)


Viewed through the lens reservoir computing approaches, we have also demonstrated that when single-delay dynamics are utilised, a learnable map in a reservoir computing network certainly exists. Moreover, such a driven system within a reservoir computing network enjoys certain proven qualities pertaining to the robustness w.r.t input- and parameter-perturbation. Accordingly, we are able to preserve the quality of a representation that would otherwise remain sensitive to various fluctuations, rendering the learning of the system’s dynamics wholly unpredictable. In addition, we obtain the robustness of the system against external noise, a phenomenon which is observed for the Fractal Dream Attractor.

Finally, we consider some physical systems to observe the methodology in practice and verify that we can indeed predict the future evolution of chaotic systems such as the double pendulum. Empirical evidence shows that a linearity measure like a Pearson correlation coefficient points at reduced functional complexity while learning the dynamics in the state space of the driven system


Considered with a philosophical-scientific lens, this project has aimed to establish a juncture between rationalism (the theory of nonautonomous dynamical systems) and empiricism (machine learning) to yield results profitable to the discipline of Applied Mathematics. This was done by showing that dynamics in an observed space can be established as topologically conjugate to the underlying system. As a result of the topological conjugacy, deep learning techniques are then subsequently applied to physically such equations from the data. In so doing, we acquire models of high fidelity that provide exceptional forecasting results on chaotic data that exhibit both statistical and topological consistency in the long run.



\end{document}